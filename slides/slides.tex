\documentclass{beamer}
\setbeamertemplate{navigation symbols}{}

\usepackage{tikz}
\usepackage{listings}

\usetheme{Warsaw}

\beamersetuncovermixins{\opaqueness<1>{25}}{\opaqueness<2->{15}}
\begin{document}
\title{LexBFS and its applications}  
\author{Guillaume Aubian}
\date{\today} 


\begin{frame}
\titlepage
\end{frame}

\begin{frame}\frametitle{Overview}\tableofcontents
\end{frame} 


\section{Graph Searches} 
\begin{frame}\frametitle{What's in a graph ?}
    \begin{block}{Graph}
	We consider non-oriented, simple and \textbf{connected} graphs
    \end{block}
    INCLUDE GRAPHS EXAMPLES AND COUNTEREXAMPLES
\end{frame}

\begin{frame}[fragile]\frametitle{Generic Search}
    \begin{lstlisting}[language = Python]
    for i in [1, ..., n]:
        if i == 1:
	    u = any vertex
	else:
	    u = any unvisited marked vertex
	visit(u)
	for v in neighbours(u):
	    mark(v)
    \end{lstlisting}
\end{frame}

\begin{frame}\frametitle{Another Characterization}
    Let's number vertices in the order they are visited.
    \begin{theorem}
        An order $\sigma$ corresponds to a Generic Search if and only if
	    
	    $$\forall a <_{\sigma} b <_{\sigma} c, ac \in E\text{ and }ab \notin E, \exists d <_{\sigma} b\text{ st }db \in E$$
    \end{theorem}
    \begin{center}
    \begin{tikzpicture}
	\draw (0,1) arc (120:60:4) ;
	\draw [dashed] (0,1) -- (2,1);
	\draw (0,0) -- (2,1);
	\draw (0,0) node[below]{$d$} ;
	\draw (0,1) node[below]{$a$} ;
	\draw (2,1) node[below]{$b$} ;
	\draw (4,1) node[below]{$c$} ;
	\draw [dotted] (1,-0.5) -- (1,1.5);
	\draw [dotted] (3,-0.5) -- (3,1.5);
	\draw (1,-0.5) node[below]{$<$} ;
	\draw (3,-0.5) node[below]{$<$} ;
    \end{tikzpicture}
    \end{center}
\end{frame}

\begin{frame}[fragile]\frametitle{DFS}
    INCLUDE GRAPH EXAMPLE
    \begin{lstlisting}[language = Python]
    for i in [1, ..., n]:
        if i == 1:
	    u = any vertex
	else:
	    u = any unvisited vertex w/ max label
	visit(u)
	for v in neighbours(u):
	    label[v] = i
    \end{lstlisting}


\end{frame}

\begin{frame}\frametitle{Another Characterization}
    Let's number vertices in the order they are visited.
    \begin{theorem}
       An order $\sigma$ corresponds to a DFS if and only if
    \end{theorem}
    INCLUDE DRAWING
\end{frame}


\begin{frame}[fragile]\frametitle{BFS}
    \begin{lstlisting}[language = Python]
    for i in [n, ..., 1]:
        if i == n:
	    u = any vertex
	else:
	    u = any unvisited vertex w/ max label
	visit(u)
	for v in neighbours(u):
	    if v has no label:
	        label[v] = i
    \end{lstlisting}


    INCLUDE GRAPH EXAMPLE
\end{frame}

\begin{frame}\frametitle{Another Characterization}
    Let's number vertices in the order they are visited.
    \begin{theorem}
       An order $\sigma$ corresponds to a BFS if and only if
    \end{theorem}
    INCLUDE DRAWING
\end{frame}

\begin{frame}[fragile]\frametitle{Let's rewrite BFS}

    \begin{lstlisting}[language = Python]
    for i in [n, ..., 1]:
        if i == n:
	    u = any vertex
	else:
	    u = any unvisited vertex
	        w/ max first element of label
	visit(u)
	for v in neighbours(u):
	    label[v].append(i)
    \end{lstlisting}

    INCLUDE GRAPH EXAMPLE
\end{frame}

\begin{frame}[fragile]\frametitle{Here is LexBFS}
    \begin{lstlisting}[language = Python]
    for i in [n, ..., 1]:
        if i == n:
	    u = any vertex
	else:
	    u = any unvisited vertex
	        w/ max lexicographical label
	visit(u)
	for v in neighbours(u):
	    label[v].append(i)
    \end{lstlisting}

	
    INCLUDE GRAPH EXAMPLE
\end{frame}

\begin{frame}\frametitle{Another Characterization}
    Let's number vertices in the order they are visited.
    \begin{theorem}
       An order $\sigma$ corresponds to a LexBFS if and only if
    \end{theorem}
    INCLUDE DRAWING
\end{frame}

\begin{frame}\frametitle{blocs}

\begin{block}{title of the bloc}
bloc text
\end{block}

\begin{exampleblock}{title of the bloc}
bloc text
\end{exampleblock}


\begin{alertblock}{title of the bloc}
bloc text
\end{alertblock}
\end{frame}

\section{Section no. 5}
\subsection{split screen}

\begin{frame}\frametitle{splitting screen}
\begin{columns}
\begin{column}{5cm}
\begin{itemize}
\item Beamer 
\item Beamer Class 
\item Beamer Class Latex 
\end{itemize}
\end{column}
\begin{column}{5cm}
\begin{tabular}{|c|c|}
\hline
\textbf{Instructor} & \textbf{Title} \\
\hline
Sascha Frank &  \LaTeX \ Course 1 \\
\hline
Sascha Frank &  Course serial  \\
\hline
\end{tabular}
\end{column}
\end{columns}
\end{frame}

%\subsection{Pictures} 
%\begin{frame}\frametitle{pictures in latex beamer class}
%\begin{figure}
%\includegraphics[scale=0.5]{PIC1} 
%\caption{show an example picture}
%\end{figure}
%\end{frame}

\subsection{joining picture and lists} 

%\begin{frame}
%\frametitle{pictures and lists in beamer class}
%\begin{columns}
%\begin{column}{5cm}
%\begin{itemize}
%\item<1-> subject 1
%\item<3-> subject 2
%\item<5-> subject 3
%\end{itemize}
%\vspace{3cm} 
%\end{column}
%\begin{column}{5cm}
%\begin{overprint}
%\includegraphics<2>{PIC1}
%\includegraphics<4>{PIC2}
%\includegraphics<6>{PIC3}
%\end{overprint}
%\end{column}
%\end{columns}
%\end{frame}



\end{document}
