\documentclass{beamer}
\setbeamertemplate{navigation symbols}{}


\usetheme{Warsaw}

\beamersetuncovermixins{\opaqueness<1>{25}}{\opaqueness<2->{15}}
\begin{document}
\title{LexBFS and its applications}  
\author{Guillaume Aubian}
\date{\today} 


\begin{frame}
\titlepage
\end{frame}

\begin{frame}\frametitle{Overview}\tableofcontents
\end{frame} 


\section{Graph Searches} 
\begin{frame}\frametitle{What's in a graph ?}
    \begin{block}
	We consider non-oriented, simple and \textbf{connected} graphs
    \end{block}
    INCLUDE GRAPHS EXAMPLES AND COUNTEREXAMPLES
\end{frame}

\begin{frame}\frametitle{Generic Search}
    INCLUDE GRAPH EXAMPLE
    \begin{definition}
    for i in {1, …, n}
	let u an unvisited marked vertex (any vertex if i = 1)
	visit u
	for v in neighbours(u)
	    mark v
    \end{definition}
\end{frame}

\begin{frame}\frametitle{Another Characterization}
    Let's number vertices in the order they are visited.
    \begin{theorem}
        An order $\sigma$ corresponds to a Generic Search if and only if
	    
	    $$\forall a <_{\sigma} b <_{\sigma} c, ac \in E\text{ and }ab \notin E, \exists d <_{\sigma} b\text{ st }db \in E$$
    \end{theorem}
    INCLUDE DRAWING
\end{frame}

\begin{frame}\frametitle{DFS}
    INCLUDE GRAPH EXAMPLE
    \begin{definition}
    for i in {1, …, n}
	u an unvisited vertex with maximum label (any vertex if i = 1)
	visit u
	for v in neighbours(u)
	   label[v] = i
    \end{definition}
\end{frame}

\begin{frame}\frametitle{Another Characterization}
    Let's number vertices in the order they are visited.
    \begin{theorem}
       An order $\sigma$ corresponds to a DFS if and only if
    \end{theorem}
    INCLUDE DRAWING
\end{frame}


\begin{frame}\frametitle{BFS}
    \begin{definition}
    for i in {n, …, 1}
	u an unvisited vertex with maximum label (any vertex if i = n)
	visit u
	for v in neighbours(u)
	   if v has no label
	       label[v] = i
    \end{definition}
    INCLUDE GRAPH EXAMPLE
\end{frame}

\begin{frame}\frametitle{Another Characterization}
    Let's number vertices in the order they are visited.
    \begin{theorem}
       An order $\sigma$ corresponds to a BFS if and only if
    \end{theorem}
    INCLUDE DRAWING
\end{frame}

\begin{frame}\frametitle{Let's rewrite BFS}
    \begin{definition}
    for i in {n, …, 1}
	let u an unvisited vertex with maximum first element of the label (any vertex if i = n)
	visit u
	for v in neighbours(u)
	   append i to label[v]
    \end{definition}
    INCLUDE GRAPH EXAMPLE
\end{frame}

\begin{frame}\frametitle{Here is LexBFS}
    \begin{definition}
    for i in {n, …, 1}
	let u an unvisited vertex with lexicographical maximum label
	visit u
	for v in neighbours(u)
	   append i to label[v]
    \end{definition}
    INCLUDE GRAPH EXAMPLE
\end{frame}

\begin{frame}\frametitle{Another Characterization}
    Let's number vertices in the order they are visited.
    \begin{theorem}
       An order $\sigma$ corresponds to a LexBFS if and only if
    \end{theorem}
    INCLUDE DRAWING
\end{frame}

\begin{frame}\frametitle{blocs}

\begin{block}{title of the bloc}
bloc text
\end{block}

\begin{exampleblock}{title of the bloc}
bloc text
\end{exampleblock}


\begin{alertblock}{title of the bloc}
bloc text
\end{alertblock}
\end{frame}

\section{Section no. 5}
\subsection{split screen}

\begin{frame}\frametitle{splitting screen}
\begin{columns}
\begin{column}{5cm}
\begin{itemize}
\item Beamer 
\item Beamer Class 
\item Beamer Class Latex 
\end{itemize}
\end{column}
\begin{column}{5cm}
\begin{tabular}{|c|c|}
\hline
\textbf{Instructor} & \textbf{Title} \\
\hline
Sascha Frank &  \LaTeX \ Course 1 \\
\hline
Sascha Frank &  Course serial  \\
\hline
\end{tabular}
\end{column}
\end{columns}
\end{frame}

%\subsection{Pictures} 
%\begin{frame}\frametitle{pictures in latex beamer class}
%\begin{figure}
%\includegraphics[scale=0.5]{PIC1} 
%\caption{show an example picture}
%\end{figure}
%\end{frame}

\subsection{joining picture and lists} 

%\begin{frame}
%\frametitle{pictures and lists in beamer class}
%\begin{columns}
%\begin{column}{5cm}
%\begin{itemize}
%\item<1-> subject 1
%\item<3-> subject 2
%\item<5-> subject 3
%\end{itemize}
%\vspace{3cm} 
%\end{column}
%\begin{column}{5cm}
%\begin{overprint}
%\includegraphics<2>{PIC1}
%\includegraphics<4>{PIC2}
%\includegraphics<6>{PIC3}
%\end{overprint}
%\end{column}
%\end{columns}
%\end{frame}



\end{document}
